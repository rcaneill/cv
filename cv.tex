\documentclass[11pt,a4paper,roman]{moderncv}

% moderncv themes
\moderncvstyle{classic}
\moderncvcolor{blue}

\usepackage{tabularx}

\usepackage{ragged2e}

\newcommand*{\cvtripleitem}[6][.25em]{
 \cvitem[#1]{#2}{
   \begin{minipage}[t]{\doubleitemmaincolumnwidth}#3\end{minipage}
   \hfill
   \begin{minipage}[t]{\hintscolumnwidth}\raggedleft\hintstyle{#4}\end{minipage}
   \hspace*{\separatorcolumnwidth}
   \begin{minipage}[t]{\doubleitemmaincolumnwidth}\raggedright\hintstyle{#5}\end{minipage}}
   \par\noindent
   \begin{minipage}[t]{\linewidth}\centering#6\end{minipage}}

% character encoding
\usepackage[T1]{fontenc}
\usepackage{lmodern}
% adjust the page margins
\usepackage[scale=0.82, top=1.5cm, bottom=1.7cm]{geometry} %scale=0.75
%\setlength{\hintscolumnwidth}{2.5cm} % if you want to change the width of the column with the dates
\recomputelengths % seems to be necessary
\include{commons/colors}

% personal data
\name{Romain}{Caneill}
%\title{PhD application} % optional
%\phone[mobile]{} % optional
\email{romain.caneill@gu.se} % optional
\address{}{41322 Gothenburg, Sweden} % optional
%\extrainfo{}
\social[github]{rcaneill}
\social[mastodon]{fediscience.org/web/@rcaneill}

% to show numerical labels in the bibliography (default is to show no labels); only useful if you make citations in your resume
%\makeatletter
%\renewcommand*{\bibliographyitemlabel}{\@biblabel{\arabic{enumiv}}}
%\makeatother
%\renewcommand*{\bibliographyitemlabel}{[\arabic{enumiv}]}% CONSIDER REPLACING THE ABOVE BY THIS

% bibliography with multiple entries
%\usepackage{multibib}
%\newcites{book,misc}{{Books},{Others}}
%----------------------------------------------------------------------------------
%            content
%----------------------------------------------------------------------------------

\begin{document}

\makecvtitle

%\vspace*{-1.2cm}

\section{Education}
\cventry{2018 -- present}
        {PhD studies}
        {Department of Marine Sciences, University of Gothenburg}
        {Gothenburg, Sweden}
        {supervised by Fabien Roquet}
        {Physical Oceanography}
\cventry{2015 -- 2017}
        {Master's degree in geophysics}
        {École Normale Supérieure de Lyon (ENSL) and Lyon 1 University}
        {Lyon}
        {France}
        {Physics and chemistry of the Earth and Planets}
\cventry{2014 -- 2015}
        {Bachelor's degree in fundamental physics}
        {École Normale Supérieure de Lyon and Lyon 1 University}
        {Lyon}{France}{}
\cventry{2012 -- 2014}
        {Two-year university degree in physics and mathematics}
        {Paris-Sud University}
        {Orsay}
        {France}
        {}


\section{Academic experience}
\cventry{2017}
        {Master's internship}
        {Institute for Geosciences and Environmental research}
        {Grenoble, France}
        {supervised by Ghislain Picard, 5 months}
        {\textbf{Analysis of Elevation Maps Measured by Laser Scannering at
            Dome C, Antarctica. Investigation of Snow Accumulation Processes.}}
\cventry{2016}
        {Master's internship}
        {Meteorological Institute of Stockholm University}
        {Stockholm, Sweden}
        {supervised by Fabien Roquet, 3 months}
        {\textbf{Assessing the fine-scale structure of the ocean
            circulation above the Kerguelen Plateau from instrumented seals.}}
\cventry{2015}
        {Bachelor's internship}
        {CEN: Snow Research Centre}
        {Grenoble, France}
        {supervised by Frédéric Flin, 2 months}
        {\textbf{Calibration of a Cryogenic Cell for the Study of Snow.}}

        
\section{Teaching experience}
\cventry{2022}
        {CodeRefinery workshop}
        {Hosted by \href{https://coderefinery.org}{coderefinery.org}}
        {6 half days online -- Git version control, testing code, produce reproducible science}
        {Exercise leader, \href{https://coderefinery.github.io/2022-03-22-workshop}{coderefinery.github.io/2022-03-22-workshop}}
        {}
\cventry{2020 -- 2022}
        {Physical oceanography II (OCM210), Master course}
        {Department of Marine Sciences, University of Gothenburg}
        {Gothenburg, Sweden}
        {Teacher assistant}
        {}
\cventry{2019}
        {Ocean modelling (OC6310), Master course}
        {Department of Marine Sciences, University of Gothenburg}
        {Gothenburg, Sweden}
        {Teacher assistant}
        {}
\cventry{2018 -- 2022}
        {Physical oceanography I (OCM100), Master course}
        {Department of Marine Sciences, University of Gothenburg}
        {Gothenburg, Sweden}
        {Teacher assistant}
        {}


\nocite{*}
\bibliographystyle{IEEEtran}
\bibliography{publications}

        
\section{Skills}

\cvlistitem{Dynamical and descriptive physical oceanography}
\cvlistitem{Climate and ocean data analysis}
\cvlistitem{Ocean modelling (NEMO)}
\cvlistitem{Profiles analyzes (e.g. ARGO)}

\subsection{Computer science}
\cvlistitem{
  Proficient with \textbf{Python} data analysis (numpy, xarray, matplotlib, jupyter), and \textbf{Snakemake}
  }
\cvlistitem{
  Good level in open- and reproducible-science practices
  (\textbf{Git}, \textbf{GitHub}, \textbf{GitHub actions}, \textbf{GNU/Linux}, \textbf{Apptainer} container system, python application testing with \textbf{pytest})
  }
\cvlistitem{
  Proficient with articles and documents production (regular practice of
  \textbf{Latex}, \textbf{Markdown})
  }
\cvlistitem{
  Basic level in \textbf{C}, \textbf{Fortran}
}

\subsection{Languages}

\cvitem{}{\textbf{French}\hspace{\separatorcolumnwidth}Native language\hfill
  \textbf{English}\hspace{\separatorcolumnwidth}Oral and written practice}

\section{Participation to open-source projects}
\cvline{\textbf{xnemogcm}}{
  Core developer. xnemogcm is a python package that opens NEMO outputs and make them compliant with xgcm.
}
\cvline{}{\textit{
    \href{https://github.com/rcaneill/xnemogcm}
         {github.com/rcaneill/xnemogcm}
}}

\cvline{\textbf{xnemogcm test data}}{
  Containers that can run NEMO 3.6 to 4.2 test case in a reproducible way to produce test data for xnemogcm.
}
\cvline{}{\textit{
    \href{https://github.com/rcaneill/xnemogcm_test_data}
         {github.com/rcaneill/xnemogcm\_test\_data}
}}

\cvline{\textbf{xgcm}}{
  I wrote the NEMO example, the documentation on grid boundary conditions, and I participated to discussions about vertical remapping.}
\cvline{}{\textit{
  \href{https://github.com/xgcm/xgcm/commits?author=rcaneill}
       {https://github.com/xgcm/xgcm/commits?author=rcaneill}
}}

\cvline{\textbf{gsw-xarray}}{
  Core developer. gsw-xarray is a xarray wrapper for gsw that adds CF attributes.
}
\cvline{}{\textit{
  \href{https://github.com/DocOtak/gsw-xarray}
       {github.com/DocOtak/gsw-xarray}
}}
\section{Hobbies}
\cvitem{\textbf{Music}}{Guitar, composition, concert sound system installation.}
\cvitem{\textbf{Sports}}{Mountaineering, climbing in competition, skiing,
  trekking.}

\end{document}
