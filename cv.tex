\documentclass[11pt,a4paper,roman]{moderncv}

% moderncv themes
\moderncvstyle{classic}
\moderncvcolor{blue}

\usepackage{tabularx}

\usepackage{ragged2e}

\newcommand*{\cvtripleitem}[6][.25em]{
 \cvitem[#1]{#2}{
   \begin{minipage}[t]{\doubleitemmaincolumnwidth}#3\end{minipage}
   \hfill
   \begin{minipage}[t]{\hintscolumnwidth}\raggedleft\hintstyle{#4}\end{minipage}
   \hspace*{\separatorcolumnwidth}
   \begin{minipage}[t]{\doubleitemmaincolumnwidth}\raggedright\hintstyle{#5}\end{minipage}}
   \par\noindent
   \begin{minipage}[t]{\linewidth}\centering#6\end{minipage}}

% character encoding
\usepackage[T1]{fontenc}
\usepackage{lmodern}
% adjust the page margins
\usepackage[scale=0.82, top=1.5cm, bottom=1.7cm]{geometry} %scale=0.75
%\setlength{\hintscolumnwidth}{2.5cm} % if you want to change the width of the column with the dates
\recomputelengths % seems to be necessary
\include{commons/colors}

% personal data
\name{Romain}{Caneill}
%\title{PhD application} % optional
%\phone[mobile]{} % optional
\email{romain.caneill@univ-grenoble-alpes.fr} % optional
\address{}{Grenoble, France} % optional
%\social[github]{rcaneill}
%\social[gitlab][gitlab.com/rcaneill/]{rcaneill}
\extrainfo{\href{https://romaincaneill.fr}{romaincaneill.fr}\\
PGP fingerprint:\\
  \texttt{70D5 7116 37B2 9335 9088}\\
  \texttt{F124 D0FE 114E BFFD ED7F}}

% to show numerical labels in the bibliography (default is to show no labels); only useful if you make citations in your resume
%\makeatletter
%\renewcommand*{\bibliographyitemlabel}{\@biblabel{\arabic{enumiv}}}
%\makeatother
%\renewcommand*{\bibliographyitemlabel}{[\arabic{enumiv}]}% CONSIDER REPLACING THE ABOVE BY THIS

% bibliography with multiple entries
%\usepackage{multibib}
%\newcites{book,misc}{{Books},{Others}}
%----------------------------------------------------------------------------------
%            content
%----------------------------------------------------------------------------------

\begin{document}


%----------------------------------------------------------------------------------------
%	COVER LETTER
%----------------------------------------------------------------------------------------

% To remove the cover letter, comment out this entire block

\clearpage

\opening{Dear Sir or Madam,} % Opening greeting
\closing{Sincerely yours,} % Closing phrase
\enclosure[Attached]{curriculum vit\ae{}} % List of enclosed documents

~\hfill \today

\vspace{1.5cm}

Dear NEMO system team,

\vspace{1.5cm}

I am contacting you to apply to the NEMO Demonstrator Hackathon that will be held from the
15th to the 21st of June in Exeter.


\vspace{.5cm}

During my PhD, I gained experience with NEMO
as we developed a single basin configuration.
In this configuration we employed terrain-following coordinates
and analytical fields for the bathymetry and forcings.


\vspace{.5cm}

I have experience in writing simple but well-designed analyses
with the \emph{Python} language. I also develop the \emph{xnemogcm} Python
library and containers containing a compiled NEMO environment.
Last, I wrote documentation for several projects,
e.g. \emph{xgcm}, \emph{gsw-xarray}.



\vspace{.5cm}


By participating in the hackathon, I hope to fully learn how to set
up a realistic configuration with sea ice.
This knowledge would be directly applicable for me,
as I currently work on the dynamics of sea ice and its modelling.
Meeting all the system team would be a great chance for potential collaboration,
as I will use again NEMO in the near future.

\vspace{.5cm}

I would like to create a demonstrator for a realistic configuration,
with a preference in the Arctic region. 

\vspace{.5cm}

\makeletterclosing % Print letter signature

\newpage



\vspace{-0.5cm}

\makecvtitle

\vspace*{-0.5cm}

\section{Academic experience -- employement}
\cventry{2024 -- 2027}
        {Postdoc}
        {Institute of Environmental Geosciences}
        {Grenoble, France}
        {working with Pierre Rampal, SASIP project}
        {Lagrangian perspective on the sea ice dynamics and turbulence}
\cventry{2018 -- 2024}
        {PhD studies}
        {Department of Marine Sciences, University of Gothenburg}
        {Gothenburg, Sweden}
        {supervised by Fabien Roquet}
        {Physical oceanography.
          \textbf{From alpha to beta ocean: Exploring the role of surface buoyancy fluxes and seawater thermal expansion in setting the upper ocean stratification}
        }


\section{Education}
\cventry{2015 -- 2017}
        {Master's degree in geophysics}
        {École Normale Supérieure de Lyon (ENSL) and Lyon 1 University}
        {Lyon}
        {France}
        {Physics and chemistry of the Earth and Planets}
\cventry{2014 -- 2015}
        {Bachelor's degree in fundamental physics}
        {École Normale Supérieure de Lyon and Lyon 1 University}
        {Lyon}{France}{}


        
\section{Teaching experience}
\cventry{2022}
        {CodeRefinery workshop}
        {Hosted by \href{https://coderefinery.org}{coderefinery.org}}
        {6 half days online -- Git version control, testing code, produce reproducible science}
        {Exercise leader, \href{https://coderefinery.github.io/2022-03-22-workshop}{coderefinery.github.io/2022-03-22-workshop}}
        {}
\cventry{2018 -- 2022}
        {Physical oceanography I and II, Master course}
        {Department of Marine Sciences, University of Gothenburg}
        {Gothenburg, Sweden}
        {Teacher assistant}
        {}
\cventry{2019}
        {Ocean modelling, Master course}
        {Department of Marine Sciences, University of Gothenburg}
        {Gothenburg, Sweden}
        {Teacher assistant}
        {}

\nocite{*}
\bibliographystyle{IEEEtran}
\bibliography{publications}

\section{Conferences -- workshops}
\cventry{2023}
        {IUGG 2023 Berlin}
        {Poster}
        {Investigating surface buoyancy flux and Ekman transport influence on the Southern Ocean's upper ocean pycnocline stratification}
        {\textbf{R. Caneill}, F. Roquet, G. Madec, J. Nycander}
        {\href{https://romaincaneill.fr/news/2023/07/12/iugg23.html}{https://romaincaneill.fr/news/2023/07/12/iugg23.html}}
\cventry{2023}
        {EGU 23}
        {Presentation}
        {The Influence of Surface Buoyancy Flux and Ekman Transport on Upper Ocean Pycnocline Stratification in the Southern Ocean}
        {\textbf{R. Caneill}, F. Roquet, G. Madec, J. Nycander}
        {\href{https://doi.org/10.5194/egusphere-egu23-11655}{https://doi.org/10.5194/egusphere-egu23-11655}}
\cventry{2022}
        {Workshop in Bornö (Sweden)}
        {Title of the workshop:
        Drivers of the global overturning circulation: wind versus buoyancy}
        {Organized by Fabien Roquet}
        {}{}
\cventry{2021}
        {virtual EGU 21}
        {vPICO}
        {What determines the position of the transition zone between alpha and beta regions in the ocean? A model study}
        {\textbf{R. Caneill}, F. Roquet, G. Madec, J. Nycander}
        {\href{https://doi.org/10.5194/egusphere-egu21-14331}{https://doi.org/10.5194/egusphere-egu21-14331}}
\cventry{2020}
        {DRAKKAR meeting}
        {Poster}
        {Sensitivity of Oceanic Fronts to Nonlinearities of Equation of State Investigated Using Numerical Experiments}
        {\textbf{R. Caneill}, F. Roquet, G. Madec, J. Nycander}
        {\href{https://github.com/rcaneill/DRAKKAR-2020-poster/blob/master/poster.pdf}{https://github.com/rcaneill/DRAKKAR-2020-poster/blob/master/poster.pdf}}

\section{Skills}

\cvlistitem{Dynamical and descriptive physical oceanography}
\cvlistitem{Climate and ocean data analysis}
\cvlistitem{Ocean modelling (run NEMO on HPC, analyze models outputs)}
\cvlistitem{Profiles analyzes (e.g. ARGO)}

\subsection{Computer science}
\cvlistitem{
  Proficient with \textbf{Python} data analysis (numpy, xarray, matplotlib, jupyter, xgcm), and \textbf{Snakemake}
  }
\cvlistitem{
  Good level in open- and reproducible-science practices
  (\textbf{Git}, \textbf{GitHub}, \textbf{GitHub actions}, \textbf{GNU/Linux}, \textbf{Apptainer} container system, python application testing with \textbf{pytest}), \textbf{Snakemake}, basics level in \textbf{GNU make}
  }
\cvlistitem{
  Proficient with articles and documents production
  (e.g. \textbf{Latex}, \textbf{Markdown}, \textbf{HTML})
  }
\cvlistitem{
  Basic level in \textbf{C}, \textbf{C++}, \textbf{Fortran}
}

\subsection{Languages}


\cvitem{\textbf{French}}{Native language}
\cvitem{\textbf{English}}{Oral and written practice}

\section{Participation to open-source projects}
\cvline{\textbf{xnemogcm}}{
  Core developer. xnemogcm is a python package that opens NEMO outputs and make them compliant with xgcm.
}
\cvline{}{\textit{
    \href{https://github.com/rcaneill/xnemogcm}
         {github.com/rcaneill/xnemogcm}
}}

\cvline{\textbf{xnemogcm-test-data}}{
  Containers that can run NEMO 3.6 to 5 test case in a reproducible way to produce test data for xnemogcm.
}
\cvline{}{\textit{
    \href{https://github.com/rcaneill/xnemogcm_test_data}
         {github.com/rcaneill/xnemogcm\_test\_data}
}}

\cvline{\textbf{xgcm}}{
  I wrote the NEMO example, the documentation on grid boundary conditions, and I participated to discussions about vertical remapping.}
\cvline{}{\textit{
  \href{https://github.com/xgcm/xgcm/commits?author=rcaneill}
       {github.com/xgcm/xgcm/commits?author=rcaneill}
}}

\cvline{\textbf{gsw-xarray}}{
  Core developer. gsw-xarray is a xarray wrapper for gsw that adds CF attributes.
}
\cvline{}{\textit{
  \href{https://github.com/DocOtak/gsw-xarray}
       {github.com/DocOtak/gsw-xarray}
}}

%% \section{Hobbies}
%% \cvitem{\textbf{Carpentry}}{Professional French Diploma in cabinet making (in French: \emph{CAP de menuiserie}).}
%% \cvitem{\textbf{Music}}{Guitar, composition, concert sound system installation.}
%% \cvitem{\textbf{Sports}}{Mountaineering, climbing, skiing,
%%   trekking, wave surfing, kite surfing.}

\end{document}
